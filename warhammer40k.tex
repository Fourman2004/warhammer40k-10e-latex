\documentclass{report}
\usepackage{wallpaper} %Used for the wallpaper
\usepackage[table]{xcolor}
\usepackage{graphicx}
\usepackage{sectsty}
\usepackage[export]{adjustbox}
\usepackage{blindtext} % Used for Lorem Ipsum
\usepackage[margin=2cm]{geometry} % Easy way to define margins
\usepackage{fancyhdr} % Fancy header and footer
\pagestyle{fancy}
\usepackage{parskip} % Package to separate paragraphs with a line instead of indentation
\usepackage{multicol} % Needed for three columns.
\usepackage[uppercase]{titlesec} % Used for easy manipulation of section headings.
\usepackage{ifxetex}
\ifxetex
\usepackage{xltxtra} % This package loads fixltx2e, metalogo, xunicode, fontspec
\newfontfamily\ork{Conduit}
\setmainfont[Ligatures={Common,TeX}]{Conduit ITC}
\else
\usepackage[scaled]{helvet} % Next three lines choose Helvetica as the font.
\renewcommand*\familydefault{\sfdefault} %% Only if the base font of the document is to be sans serif
\usepackage[T1]{fontenc}
\fi


\ifxetex
\titleformat*{\chapter}{\centering\ork\thispagestyle{fancy}}
\else
\titleformat*{\chapter}{\centering\thispagestyle{fancy}}
\fi

\titlespacing\chapter{0pt}{-4ex}{\parskip} % Removing spacing between section headings and their paragraphs
\titlespacing\section{0pt}{0ex}{-\parskip} % Removing spacing between section headings and their paragraphs
\titlespacing\subsection{0pt}{0ex}{-\parskip}
\titlespacing\subsubsection{0pt}{0ex}{-\parskip}

\setcounter{secnumdepth}{-1} % Get rid of all numbering, but let everything turn up in the table of contents


\renewcommand{\headrule}{\includegraphics[width =1.13\textwidth]{Images/Header_footer/Header.png}} % Fancy ork header.
\renewcommand{\footrule}{\includegraphics[width =1.13\textwidth, height=2.45cm]{Images/Header_footer/Footer.png}}

%\makeatletter % Anything between this and \makeatother would usually live in a class i.e. commands meant for only a cls file.
%\renewcommand\@maketitle{\centering\Huge\@title\\} % Casually redefining how the title works. NB, this will make twocolumn titles only live in one column.
%\makeatother


\begin{document}
\color[HTML]{FFFFFF}\chapter{Detachment name here}
\textbf{Last Editied: \today}
%\thispagestyle{fancy} % Needed to make sure we get the nice header even on the ``titlepage''
\begin{multicols}{2}
\LRCornerWallPaper{1}{background.png}
% Replace "Detachment Rule" with The name of the detachment
\color[HTML]{000000}\section{Detachment Rule}\label{sec:rule}%
\textit{Fluff Here.}\\
Detachment rule here\\
\begin{adjustbox}{width=\columnwidth}
\begin{tabular}{|
>{\columncolor[HTML]{3f3f3f}}l |l|}
\hline
Battle Size & Resource\\ \hline
{\color[HTML]{FFFFFF} Combat Patrol} & Xpts \\ \hline
{\color[HTML]{FFFFFF} Incursion}     & Ypts\\ \hline
{\color[HTML]{FFFFFF}  Strike Force}  & Zpts \\ \hline
{\color[HTML]{FFFFFF} Onslaught}     & Wpts\\ \hline
\end{tabular}
\end{adjustbox}
% Please add the following required packages to your document preamble:
% \usepackage[table,xcdraw]{xcolor}
% Beamer presentation requires \usepackage{colortbl} instead of \usepackage[table,xcdraw]{xcolor}
\vfill\null
\columnbreak

%This table is for any mechanic that relies on a select amount of points or units.
\color[HTML]{000000}\section{Enhancements}
\vspace{1mm}
\subsection{Enhancement name}
\vspace{1mm}
\textit{Fluff here.}\\
\textbf{Model here.} Rules here\\
\subsection{Enhancement name}
\vspace{1mm}
\textit{Fluff here.}\\
\textbf{Model here.} Rules here\\
\subsection{Enhancement name}
\vspace{1mm}
\textit{Fluff here.}\\
\textbf{Model here.} Rules here\\
\subsection{Enhancement name}
\vspace{1mm}
\textit{Fluff here.}\\
\textbf{Model here.} Rules here\\
\vspace{1mm}
\begin{adjustbox}{width=\columnwidth}
\begin{tabular}{|
>{\columncolor[HTML]{3f3f3f}}l |l|}
\hline
{\color[HTML]{FFFFFF} Enhancement} & Xpts \\ \hline
{\color[HTML]{FFFFFF} Enhancement}     & Ypts\\ \hline
{\color[HTML]{FFFFFF}  Enhancement}  & Zpts \\ \hline
{\color[HTML]{FFFFFF} Enhancement}     & Wpts\\ \hline
\end{tabular}
\end{adjustbox}
\end{multicols}
\color[HTML]{FFFFFF}\section{Stratagems}
\begin{minipage}[t][23.75cm][b]{\textwidth}
% Remove background from the .psd File and Export as one of the included Graphics
\includegraphics[width =0.5\textwidth]{Images/Stratagems/Stratagem_1.png}
\includegraphics[width =0.5\textwidth]{Images/Stratagems/Stratagem_2.png}
\includegraphics[width =0.5\textwidth]{Images/Stratagems/Stratagem_3.png}
\includegraphics[width =0.5\textwidth]{Images/Stratagems/Stratagem_4.png}
\includegraphics[width =0.5\textwidth]{Images/Stratagems/Stratagem_5.png}
\includegraphics[width =0.5\textwidth]{Images/Stratagems/Stratagem_6.png}
\end{minipage}
\end{document}

%%% Local Variables: 
%%% mode: latex
%%% TeX-master: t
%%% End: 

